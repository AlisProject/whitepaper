\documentclass{jsarticle}
\begin{document}

\title{ALIS whitepaper}
\author{Masahiro Yasu and Sota Ishii and Takashi Mizusawa}
\maketitle

\section{目次}
\begin{itemize}
	\item ALISとは
	\item プラットフォームの紹介と特徴
	\item なぜ良質なコンテンツが集まるのか
	\item なぜ人々に価値を還元できるのか
	\item なぜALISは長期的な発展を続けることができるのか
	\item トークンはどのように作成・配布するのか
	\item 作成・配布のロジックはどのようなロジックなのか
	\item 不正はどうやって防ぐのか
	\item 技術的な優位性or技術面について * 入れるかどうか含めて検討
	\item どこにブロックチェーン技術を用いるのか
	\item チームのビジョン・ミッションおよびチームメンバーの詳細プロフィール
	\item 将来的なプラットフォームの展望について
	\item 人の信頼性を結果的に担保し、日本の国策に紐付けてグロースさせていきたい
	\item プラットフォームのグロース戦略
	\item ファイナンシャル(ちょこっと戦略)
	\item お金の使い道
	\item 結論
\end{itemize}
\section{ALISとは}
ALISとは、日本初の分散型ソーシャルメディアプラットフォームである。従来のメディアと異なり、人々が認めるコンテンツを多くの個人が生み出す・発掘することを可能にする全く新しいソーシャルメディアである。正直に言うと、我々はSTEEM(https://steem.io)に大きな感銘を受けたところからこのプラットフォームの構想をスタートした。具体的には我々のプロジェクトメンバーであるSota Ishiiが数枚の旅行の写真をSTEEMに掲載したことから始まる。なんとその記事はわずか一日で\$30近くの評価を得ることができ、彼は無事そのお金を使ってドミノ・ピザの最高級ピザを購入することができた。我々は強烈な衝撃を受けるとともになぜこのような仕組みが実現出来ているのかを知りたくなり、徹底的にSTEEMを調べ上げた。そしてSTEEMを調べれば調べるほどその素晴らしさを知ることになった。STEEMはプラットフォーム自体が評価されることにより、STEEMのトークン自体の価値が向上しより高値でexchangeで取引されるようになる。その価値の上昇分がコンテンツのクリエイターおよび投票者に分配されることでこのような仕組みを実現していたのである。さらに、STEEMはトークンのみですでに200億の評価を得ている。新しいメディアの時代がやってきていることを身をもって体感した出来事であった。しかしながら素晴らしい面がある一方で、STEEMには2つの欠点があるように思えた(欠点があってもなおSTEEMは素晴らしいサービスであり、我々が大きなインスピレーションを彼らから得たことには変わりない)。一点目は、プラットフォームを支えるトークンの仕組みが複雑すぎてとても理解に時間がかかること。STEEMのみならず、STEEM SP STEEM Dollarsというトークンが存在し、入手のための方法も複雑であり、これはリテラシーの低い新規ユーザを遠ざけるのに十分な理由になってしまう。二点目は、日本語対応がされていないこと。我々は日本においてSTEEMの良さを活かしつつ欠点も改良した新たなソーシャルメディアが成功することを確信し、このプロジェクトをスタートする決意を固めた。
\section{プラットフォームの紹介と特徴}
ALISは、人々が良いと思うコンテンツを作成したクリエイターおよびそのようなコンテンツに投票をした人々に対して、より多くのALISトークンを配布するという報酬により信頼できるコンテンツ・人を発掘することができるソーシャルメディアプラットフォームである。そもそもこのようなメディア自体がかなり新しい概念のものになるため、STEEMと比較をした際の我々のプラットフォームの特徴を説明すると、1.我々が利用するトークンは一つでシンプルであること2.日本向けに開発されたメディアであること3.あえて不安定性を許容し、インフレ率を抑えることで長期的なプラットフォーム維持を実現していること の3点である。それぞれの特徴については次章以降におって説明をしていく。
\section{なぜ良質なコンテンツが集まるのか}
人々が良いと思うコンテンツを作り出す、もしくは投票するために必要なことは何であろうか。それは、より多くの人が認めるコンテンツを作り出すあるいは発掘した人に多くの報酬を支払うことである。コンテンツ作成者については、自分が作成したコンテンツがより多くの人に、かつよりALISトークンを多く持つ人に投票されることでより多くの報酬を得ることが可能になる。また、良質なコンテンツの発掘者に対しても、誰よりも早くかつ多くの人が良いと認めるコンテンツに投票することでより多くの報酬を得ることができる。これらに加えて、ALISトークンを多く持っている人であるほど報酬の量は線形で増えていく。つまり、よりALISトークンの報酬を得れば得るほど良質なコンテンツを生み出す or コンテンツを発掘するインセンティブが働き、更に良質なコンテンツが集まるというグッドスパイラルを形成することができる。これが良質なコンテンツが集まる理由である。
\section{なぜ人々に価値を還元できるのか}
我々のトークンはなんでもないただの電子データであり、それ単体では価値を持ち得ない。しかしながら、ALISのプラットフォームの価値があると認められれば認められるほどALISトークンが価値を持ち、取引所で高値で取引されるようになる。一見懐疑的に思われるかもしれないが、STEEMは同スキームですでに200億超の評価を得ていることからこのスキームは実現可能であることが証明されている。また我々は基盤技術にブロックチェーンを利用しており、安価に高信頼のメディアを開発することができる。運営にコストがかからない分、より多くの価値をALISに貢献したメンバーに返すことができるということである。
\section{なぜALISは長期的な発展を続けることができるのか}
ALISが真に価値のあるプラットフォームとして認められるために最も重要なファクターがある。それは、ユーザーが粘着性を持ってずっと使い続けたいと思うプラットフォームを提供することである。そのための方法はいくつかあるが、一つのこだわりとしてトークンの性質を取り上げたい。我々のトークンは、ALISのプラットフォーム上で活用されている場合のインフレ率が30\%と高い。これは、昨今の仮想通貨が取引目的ばかりにexchangeで扱われていることに関する危機感から設定された数字である。ALISトークンを所有し、プラットフォームを発展させようと努力する人であればあるほどより多くのトークンを得ることができる仕組みづくりが重要である。しかしながら、この条件だけであればトークンをALISに預けたあと、増え次第すぐに引き出してexchangeで売るというインセンティブを防ぎ得ない。そこで我々ははXEMのPoIから着想し、トークンを移してから実効性をもつまでに時間が必要であるというロジックを導入する。具体的な式は以下である。
\begin{equation}
y = log_e(t-1) , t > 0
\end{equation}
ここでtはトークンをALIS上のwalletにうつしてから経過した時間である。上記数式を採用した理由は3点ある。1点目は、新規ユーザが早くALISの魅力に気づくことができるよう、tが小さいときには上昇幅が大きいということ。2点目は、長くユーザが使うことで100\%ALISのトークンの影響を受けることができるということ。3点目は、一度引き出してしまうとまたゼロから時間を経過させる必要があるため、簡単に引き出したくならないインセンティブをユーザに与えていることである。真にALISに貢献したいと思うユーザであればあるほどこの数式が合理性を持ち、長期的なプラットフォームの発展に貢献することができる。
\section{トークンはどのように作成・配布するのか}
ALISはICOによって資金を調達する予定であるが、preminedされた通貨になる。初期に500億枚を発行し、BticoinおよびEthereumとの交換を実施する。配布分の上限は200億枚であり、残りの300億枚については我々や我々のステークホルダーが所有することになる。我々が300億枚と全体の60\%を保有することには理由がある。一点目は我々が保有することで、我々自身がプラットフォームを発展させるという健全なインセンティブを持つということ。二点目は、状況に応じて我々が供給量を増やすことにより安定な価格を維持するということ。しかしながら、トークンの保有量を我々が最も抱えているからと言って、プラットフォームの価値を想像する決定権を我々が持っているということではないことにご留意いただきたい。あくまでも良いコンテンツを作り出し、それを発掘するのはユーザである。そのようなユーザを集めるための戦略については別の章で後述する。ALISトークンはインフレ率が30\%のトークンであるとお伝えしたが、そのインフレ分がどのように配布されるかを説明する。基本的は思想は先述のとおり2点であり、1.素晴らしいコンテンツを作ったと認められた人に配布される 2.素晴らしいと人々が認めるコンテンツにいち早く投票した人に配布されるということである。この配布量について、ALISトークンの所有量が多ければ多いほど配布量を多く受け取れるというロジックを構築する。つまり、長くALISのプラットフォームに貢献をし、トークンを多く保有する人たちを最も重要なステークホルダーと捉え、彼らが力を持つことをルールとして設定する。これはPoSの仕組みに近しいものであり、富めるものがより富む構造にあるのではないか、と疑問を抱くかもしれない。確かにそういった側面は否定できないが、そのデメリットを補って余りあるステークホルダーによる運営メリットを享受すること、および新規参入者がトークンを持っていなくてもトークンが配布されるロジックを組み入れることによりプラットフォーム全体として見たときにはプラスの影響力があると考えている。
\section{不正はどうやって防ぐのか}
\end{document}
